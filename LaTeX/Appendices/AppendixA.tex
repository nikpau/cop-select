% Appendix A

\chapter{Appendix} % Main appendix title

\label{AppendixA} % For referencing this appendix elsewhere, use \ref{AppendixA}


\section{Proof of \ref{tau-copula}}

In \ref{emp-kendall} we discussed an empirical version of Kendall's Tau. For this proof we use the population analogue, given by

\begin{equation*}
	\tau\left(X_{1}, X_{2}\right)=P\left[ \left(X_{11}-X_{21}\right)\left(X_{12}-X_{22}\right)>0\right] -P\left[ \left(X_{11}-X_{21}\right)\left(X_{12}-X_{22}\right)<0\right] ,
\end{equation*}
%
where $(X_{11},X_{12})$ and $(X_{21},X_{22})$ are independent copies of $(X_1,X_2)$.
%
Now, based on 

\begin{equation*}
	P\left[ \left(X_{11}-X_{21}\right)\left(X_{12}-X_{22}\right)>0\right] =1-P\left[ \left(X_{11}-X_{21}\right)\left(X_{12}-X_{22}\right)<0\right],
\end{equation*}
%
one can write $\tau=2 P\left(\left(X_{11}-X_{21}\right)\left(X_{12}-X_{22}\right)>0\right)-1$.
%
Furthermore, using 

\begin{equation*}
	{\scriptstyle P\left[ \left(X_{11}-X_{21}\right)\left(X_{12}-X_{22}\right)>0\right] =P\left(X_{11}>X_{21}, X_{12}>X_{22}\right)+P\left(X_{11}<X_{21}, X_{12}<X_{22}\right)}
\end{equation*}
%
and the transformation $u_{1}:=F_{1}\left(x_{1}\right)$ and $u_{2}:=F_{2}\left(x_{2}\right)$ one obtains:

\begin{equation*}
	\begin{aligned}
		P\left(X_{11}>X_{21}, X_{12}>X_{22}\right) &=P\left(X_{21}<X_{11}, X_{22}<X_{12}\right) \\
		&=\int_{-\infty}^{\infty} \int_{-\infty}^{\infty} P\left(X_{21}<x_{1}, X_{22}<x_{2}\right) d C\left(F_{1}\left(x_{1}\right), F_{2}\left(x_{2}\right)\right) \\
		&=\int_{-\infty}^{\infty} \int_{-\infty}^{\infty} C\left(F_{1}\left(x_{1}\right), F_{2}\left(x_{2}\right)\right) d C\left(F_{1}\left(x_{1}\right), F_{2}\left(x_{2}\right)\right) \\
		&=\int_{0}^{1} \int_{0}^{1} C\left(u_{1}, u_{2}\right) d C\left(u_{1}, u_{2}\right).
	\end{aligned}
\end{equation*}
%
The same can be shown for 

\begin{equation*}
		P\left(X_{11}<X_{21}, X_{12}<X_{22}\right)=\int_{0}^{1} \int_{0}^{1}\left[1-u_{1}-v_{1}+C\left(u_{1}, u_{2}\right)\right] d C\left(u_{1}, u_{2}\right),
\end{equation*}
%
and as $C$ is the distribution function of $U_j := F_j(X_j)$ for $j = 1,2$ with mean $1/2$ one obtains

\begin{equation*}
	\begin{aligned}
		P\left(X_{11}<X_{21}, X_{12}<X_{22}\right) &=1-\frac{1}{2}-\frac{1}{2}+\int_{0}^{1} \int_{0}^{1} C\left(u_{1}, u_{2}\right) d C\left(u_{1}, u_{2}\right) \\
		&=\int_{0}^{1} \int_{0}^{1} C\left(u_{1}, u_{2}\right) d C\left(u_{1}, u_{2}\right)
	\end{aligned}.
\end{equation*}
%
Therefore, 

\begin{equation}
	\tau=-1+4 \int_{[0,1]^{2}} C\left(u_{1}, u_{2}\right) \mathrm{d} C\left(u_{1}, u_{2}\right).
\end{equation}

\section{Proof of \ref{beta-copula}}

This proof is taken from \citet{genest2013copula}, but also repeated here for completeness. For two random variables $X$ and $Y$ with their respective medians $\tilde{X}$ and $\tilde{Y}$, Blomqvist's Beta is given by 

\begin{equation*}
	\beta=P\{(X-\tilde{X})(Y-\tilde{Y})>0\}-P\{(X-\tilde{X})(Y-\tilde{Y})<0\} .
\end{equation*}
%
From this we can directly show that

\begin{equation*}
	\begin{split}
		P\{(X-\tilde{X})(Y-\tilde{Y})>0\}&=P(X-\tilde{X}>0, Y-\tilde{Y}>0)+P(X-\tilde{X}<0, Y-\tilde{Y}<0) \\
		&= P(X<\tilde{X},Y<\tilde{Y}) + P(X>\tilde{X},Y>\tilde{Y}) 
	\end{split}
\end{equation*}
%
and by using the properties of the median

\begin{equation*}
	P(X>\tilde{X}, Y>\tilde{Y})=P(X<\tilde{X}, Y<\tilde{Y})
\end{equation*}
%
and

\begin{equation*}
	P\{(X-\tilde{X})(Y-\tilde{Y})>0\} = 1- P\{(X-\tilde{X})(Y-\tilde{Y})<0\}
\end{equation*}
it becomes evident that using Sklar's theorem in \ref{sklar} we archive the Copula representation

\begin{equation}
	\beta =4 C\left(\frac{1}{2}, \frac{1}{2}\right) -1
\end{equation}